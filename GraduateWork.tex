\documentclass[14pt]{extreport}
\usepackage[utf8]{inputenc}
\usepackage[english,russian]{babel}

\usepackage{array}
\usepackage{setspace}
\usepackage[pdftex]{graphicx}


\usepackage{algorithm}
\usepackage{algpseudocode}

\usepackage[nottoc,numbib]{tocbibind} %bibliography in table of contents

\usepackage{titlesec} % decrease chapter heading sizes
    \titleformat{\chapter}[display]
      {\normalfont\large\bfseries\filcenter}{\chaptertitlename\ \thechapter}{0pt}{\Large} %huge 10 Huge
    \titlespacing*{\chapter}
      {0pt}{0pt}{10pt} %10 30 20 default

\makeatletter
\renewcommand{\@biblabel}[1]{#1.} % Заменяем библиографию с квадратных скобок на точку:
\renewcommand{\baselinestretch}{1.5}
\makeatother

\usepackage[left=3.0cm,right=1.5cm,top=2.0cm,bottom=2.0cm,bindingoffset=0cm]{geometry}

\begin{document}

   \begin{titlepage}{
        \thispagestyle{empty}\newgeometry{left=2.5cm,right=1.5cm,top=0cm,bottom=0.2cm,bindingoffset=0cm}\setstretch{1}
        \begin{center}
            {\footnotesize
                Министерство образования и науки Российской Федерации\\
                Федеральное государственное автономное образовательное учреждение\\
                высшего профессионального образования\\
            }

            <<Уральский федеральный университет\\
             имени первого Президента России Б.Н.Ельцина>>

             \vskip+0.5cm

            Институт математики и компьютерный наук\\
            Кафедра вычислительной математики

            \vskip+25mm

            {\bf \LARGE
                Поиск аномалий в DNS трафике \\
            }

            \vskip+15mm
        \end{center}
	
        \vfill
        \noindent\begin{parbox}[t]{9cm}{\small
                \vspace{2.0cm}
                Допустить к защите:

                \bigskip
                \bigskip
                \bigskip

                \hbox to45mm{\hrulefill}

                \bigskip

                <<\,\hbox to10mm{\hrulefill}\,>>  \hbox to25mm{\hrulefill}  2016 г.
            }
            \end{parbox}
            \begin{parbox}[t]{9cm}{\small  \setstretch{1}
                Выпускная квалификационная работа \\
                на степень бакалавра по направлению\\
                02.03.02
                Фундаментальная информатика \\
                и информационные технологии	 \\
                студента группы МК-420002 \\
                                \bigskip
                Меньших Ивана Александровича\\
                Научный руководитель\\
                доцент Солодушкин Святослав Игоревич\\
            }
            \end{parbox}
        \vfill
        \centerline{Екатеринбург}
        \centerline{2016}
        }\restoregeometry
    \end{titlepage}
\newpage
    \tableofcontents

\newpage
    \chapter{Введение в проблему поиска вредоносной активности}

Важнейшей задачей, стоящей перед системными администраторами, является защита корпоративной сети и пресечение любых вредоносных взаимодействий клиентов сети и сети интернет. Анализ логов DNS сервера позволяет выделить в общей массе запросов <<подозрительные>>, которые могут соответствовать коммуникации зараженного хоста (клиента) и, например, управляющего C\&C сервера.

В реальности, анализ DNS логов проводить крайне трудно, потому что каждый клиент генерирует большое количество запросов. Даже чтобы зайти на сайт, пользователь генерирует несколько запросов. Как правило, при загрузке страницы происходят обращения на сторонние сервера для загрузки стилей, скриптов, изображений и т.д. Даже когда пользователь ничего не делает, его ПК генерирует запросы. Это могуть быть обновления программного обеспечения, синхронизация часов и так далее. Особенно это заметно в новых версиях Windows: операционная система старается <<радовать>> пользователя своей интерактивностью и постоянно запрашивает новости, курсы валют, отправляет различную информацию о его действиях на свои сервера для <<персонализации>> взаимодействия.

В этой работе будут рассмотрены различные способы анализа DNS трафика, на базе которых можно выделить домены, которые, вероятно, используются для коммуникации между зараженным хостом и ботмастером.

\newpage
Общая идея заключается в анализе паттернов взаимодействия зараженных клиентов и C\&C серверов. Кроме того, речь будет идти о фильтрации запросов. Это очень важный аспект, ведь известно, что <<Garbage In, Garbage Out>>. Будут сделаны несколько предположений о структуре паттернов взаимодействия, а также предложены методы анализа.

\chapter{Формальная постановка задачи}
	Для того, чтобы приступить к решению этой проблемы, необходимо формализовать задачу.
	
	{\bfДано:}
		
		1. DNS логи ($querylog$). Определим их как множество $Q$. Каждая запись в нём представима в виде вектора признаков $\vec{query}$$\in$$Q$, который \textbf{обязательно} содержит $source\_ip$ (ip адрес клиента, который совершает запрос) и $domain$ (домен, который пользователь желает <<разрешить>>). Кроме того, в нём могут содержаться дополнительные данные (например время, когда был совершен запрос, подробно структура $\vec{query}$ будет описана далее).
		
		2. Белый список доменов ($whitelist$). Будем использовать <<самые частопосещаемые>> домены, воспользуемся списками от Alexa и Quantcast)
		
		3. Черный список доменов ($blacklist$). Будем использовать базы компании SkyDNS и прочие источники.
	
	{\bfНеобходимо:} 
	
	1. Выбрать из $querylog$ подмножество 
	запросов $\{\vec{query_1}, \vec{query_2}, \vec{query_3}, \dots\}$, которые были совершены к вредоносным доменам и выделить имена этих доменов
	
	2. Выделить общие паттерны взаимодействия клиентов и вредоносных доменов, т.е. описать внешний вид графа запросов.
	
	\section{Структура $\vec{query}$}
		\begin{tabular}{| l | r |}
			\hline
			Поле & Описание \\ \hline
			$domain$ & Домен, который был запрошен пользователем \\ \hline 
			$source\_ip$ &  IP адрес пользователя \\ \hline
			$rcode$ &  Код возврата DNS сервера \\ \hline
			$qtype$ &  Код запрашиваемого типа записи \\ \hline
			$timestamp$ &  Временная метка получения сервером запроса \\ \hline

		\end{tabular}
		
	
	\section{Структура $whitelist$ и $blacklist$}
	\begin{tabular}{| l | r |}
			\hline
			Поле & Описание \\ \hline
			$domain$ & Доменное имя \\ \hline 
			$cats$ & Cписок категорий, к которым принадлежит домен \\ \hline

	\end{tabular}
	\section{Словарь}
	Тут будут всякие термины (ботнет, C\&C, etc)
	\chapter{Ботнеты}
	\section{Описание структуры}
	тут будет текст
	
	\chapter{Групповая активность ботнетов}
	Это я уже попробовал немного, работает не очень хорошо, очень много мусора даёт на выходе, но тем не менее оно работает.
	\section{Описание подхода}
	тут будет текст
	\section{Алгоритм}

\begin{algorithmic}
\If {$igeq maxval$}
    \State $igets 0$
\Else
    \If {$i+kleq maxval$}
        \State $igets i+k$
    \EndIf
\EndIf
\end{algorithmic}
	
	\chapter{Ранжирование доменов}
	Это я пробую сейчас, пока получаются странные результаты, которые я не могу объяснить, требует сборки тестового стенда
	\section{Описание подхода}
	тут будет текст
	\section{Алгоритм}
	тут будет текст

	\chapter{<<Графовый>> подход}
	Это я начал делать, упёрся в три-факторизацию матриц, буду продолжать как закончу предыдущие. Но именно этот подход выглядит как самый перспективный
	\section{Описание подхода}
	тут будет текст
	\section{Алгоритм}
	тут будет текст

	\chapter{Анализ временных рядов запросов}
	Попробовал, после преобразования Фурье получаю очень много шума, нужно придумать как чистить
	\section{Описание подхода}
	тут будет текст
	\section{Алгоритм}
	тут будет текст
	
	\chapter{Байесовский подход}
	Ещё не пробовал, ничего сказать пока не могу. Возможно не успею сделать и выброшу его совсем из диплома.
	\section{Описание подхода}
	тут будет текст
	\section{Алгоритм}
	тут будет текст

	\chapter{Результаты исследований}
	тут будет текст
	
	\chapter{Литература}
	тут будет текст
	
	

\end{document} 