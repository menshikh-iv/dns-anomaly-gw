\documentclass[12pt,pdf,hyperref={unicode}]{beamer}

% \documentclass[aspectratio=43]{beamer}
% \documentclass[aspectratio=1610]{beamer}
% \documentclass[aspectratio=169]{beamer}

\usepackage[T2A]{fontenc}
\usepackage[utf8]{inputenc}
\usepackage{tikz}
\usepackage[english,russian]{babel}
\addto{\captionsrussian}{
	\renewcommand{\figurename}{Рисунок}
}
\usepackage[labelsep=endash]{caption}



\newcommand\Fontvi{\fontsize{18}{20}\selectfont}
	
\graphicspath{{prez/}}


% отключить клавиши навигации
\setbeamertemplate{navigation symbols}{}


\usetheme{Copenhagen}
\usecolortheme{default}
\setbeamertemplate{frametitle}[default][center]
\usepackage{lmodern}
%\setbeamertemplate{footline}{}
\setbeamertemplate{footline}[text line]{%
  \parbox{\linewidth}{\vspace*{-20pt}\hfill\insertpagenumber}
}

\defbeamertemplate*{title page}{customized}[1][]
{
  \vskip+0.5cm
  \centering\usebeamerfont{institute}\insertinstitute\par
  \vskip+1.2cm
  \usebeamerfont{title}\inserttitle\par
  \vskip+0.5cm
  \small\usebeamerfont{author}\insertauthor\par
  \vskip+1.2cm
  \footnotesizeЕкатеринбург\\2016
  
  %\usebeamerfont{date}\insertdate\par
  \usebeamercolor[fg]{titlegraphic}\inserttitlegraphic
}

\title{Поиск вредоносной активности в DNS трафике}

\institute{
\tiny Министерство образования и науки Российской Федерации\\ 
Федеральное государственное автономное образовательное учреждение \\ высшего образования\\
<<Уральский федеральный университет имени \\первого Президента России Б. Н. Ельцина>>\\
\vskip+0.3cm
\scriptsizeИнститут математики и компьютерных наук\\
Кафедра вычислительной математики}

\author{
	Студент: Меньших И.А.
	\and \\
	Руководитель: к.ф.-м.н., доцент Солодушкин С.И.
}
\date{}

\begin{document}


% титульный слайд
\begin{frame}
\titlepage
\end{frame} 

\begin{frame}
\frametitle{Постановка задачи}
\begin{block}{Дано}

%Дано:
\begin{enumerate}
	\item DNS-логи

	\item Белый список (Alexa, Quancast)
	\item Черный список (SkyDNS)
\end{enumerate}
\end{block}

\begin{block}{Необходимо}
\begin{enumerate}
	\item Найти вредоносные домены
	\item Выделить общие паттерны взаимодействия клиентов и вредоносных доменов
	\item Реализовать полученные подходы в виде программного кода и внедрить результат в производство
\end{enumerate}
\end{block}
\end{frame}

\begin{frame}

\frametitle{Обзор подходов} 
\Fontvi

\begin{center}
\begin{minipage}{.69\textwidth}
\begin{enumerate}
	\item Поэтапная фильтрация
	\item Анализ pDNS и WHOIS
	\item Sandbox
\end{enumerate}
\end{minipage}
\end{center}
\end{frame}

\begin{frame}
\Fontvi

\frametitle{Методы решения} 

\begin{center}
\begin{minipage}{.72\textwidth}
\begin{enumerate}

	\item Групповая активность
	\item Ранжирование доменов
	\item Поиск и анализ паттернов взаимодействия

\end{enumerate}
\end{minipage}
\end{center}
\end{frame}
\begin{frame}
\frametitle{Групповая активность}

Предположения:
\begin{enumerate}
	\item Зараженных хостов в сети фиксированное количество.
	\item Взаимодействие зараженных хостов и C\&C сервера происходит периодически.
	\item  Подозрительно, когда один и тот же домен в разное время запрашивает узкий круг хостов.
\end{enumerate}
\end{frame}

\begin{frame}
\frametitle{Групповая активность}
\begin{block}{Алгоритм поиска групповой активности}
\vskip+0.2cm
\begin{enumerate}
	\item Делим логи на <<окна>> фиксированного размера.
	\item Группируем хосты по домену.
	\item Фильтруем по черным/белым спискам и количеству уникальных пользователей.
	\item Сравниваем группы пользователей одного и того же домена в разных <<окнах>>, если группы сильно схожи - домен подозрительный.
\end{enumerate}
\end{block}
\end{frame}

\begin{frame}
\frametitle{Ранжирование доменов}
Идея: рассмотрим запросы пользователей за некоторый промежуток времени как граф.
\newline
\begin{definition}[Граф запросов]Двудольный неориентированный граф $(H\times D,  E)$, 
где $H$ - множество хостов, $D$ - множество доменных имён, ($h_i$, $d_j$)$\in$$E$, если пользователь $h_i$ запрашивал домен $d_j$. 
\end{definition}

\begin{figure}[H]
	\centering
	\includegraphics[scale=2.]{b_graph.png}
\end{figure}
\end{frame}


\begin{frame}
\frametitle{Ранжирование доменов}
Инициализируем начальные значения, опираясь на белый/черный список и итеративно будем вычислять оценки для доменов:
\begin{equation}
	\label{eq:dr1-black}
	black\_score(h_i) = \sum_{d_j: (h_i, d_j)\in E} \frac{black\_score(d_j)}{deg(d_j)}
\end{equation}
\begin{equation}
	\label{eq:dr1-white}
	white\_score(h_i) = \sum_{d_j: (h_i, d_j)\in E} \frac{white\_score(d_j)}{deg(d_j)}
\end{equation}
\begin{equation}
	\label{eq:dr1-union}
	union\_score(d_i) = \frac{black\_score(d_i)}{black\_score(d_i) + white\_score(d_i)}
\end{equation}
\begin{equation}
	\label{eq:dr2-rank}
	rank\_score(d_j) = \sum_{h_i: (h_i, d_j)\in E} \frac{rank\_score(h_i)}{deg'(h_i)}
\end{equation}
\end{frame}


\begin{frame}
\frametitle{Ранжирование доменов}
Процесс, который скрывается за формулами 1 -- 4
\begin{figure}[H]
	\centering
	\includegraphics[scale=0.175]{ranking.png}
	\caption{Схематичный пример вычисления ранга}
\end{figure}
\end{frame}

\begin{frame}
\frametitle{Анализ паттернов взаимодействия}
Идея: будем рассматривать в графе только ребра, соответствующие неудачным запросам и искать <<плотные>> подграфы
\begin{figure}[H]
	\centering
	\includegraphics[scale=0.12]{img/fail-graph.png}
	\hskip+5cm\caption{Граф DNS-запросов (\tikz{\draw[fill=blue] (0,0) circle (0.1cm);} - хосты, \tikz{\draw[fill=red] (0,0) circle (0.1cm);} - домены)}
\end{figure}

\end{frame}



\begin{frame}
\frametitle{Анализ паттернов взаимодействия}
\begin{block}{Алгоритм декомпозиции графа}
\vskip+0.2cm
\begin{enumerate}
	\item Удаляем ребра (DNS-Overload, SERVFAIL, etc)
	\item Находим все компоненты связности (BFS)
	\item Для каждой компоненты связности:
	\begin{enumerate}[a)]
		\item Ищем плотные подграфы (3-NMF или иное)
		\item Вычисляем $$density = \frac{|E|}{|H|*|D|}$$
		\item Удаляем те подграфы, которые либо маленькие, либо имеют малый $density$
	\end{enumerate}
\end{enumerate}
\end{block}
\end{frame}

\begin{frame}
\frametitle{Анализ паттернов взаимодействия}
\begin{figure}[H]
	\centering
	\includegraphics[scale=0.2]{decompose.png}
\end{figure}
\end{frame}

\begin{frame}
\frametitle{Объединение подходов и автоматизация анализа}
\begin{figure}[H]
	\centering
	\includegraphics[scale=0.37]{schema-head.png}
\end{figure}
\end{frame}

\begin{frame}
\frametitle{Объединение подходов и автоматизация анализа}
\begin{figure}[H]
	\centering
	\includegraphics[scale=0.37]{schema-tail.png}
\end{figure}
\end{frame}

\begin{frame}
\frametitle{Результаты работы}
\begin{enumerate}
	\item Разработан программный продукт, который занимается поиском вредоносных доменов.
	\item Продукт внедрен в эксплуатацию в компании SkyDNS и показывает хорошие результаты в реальных условиях.
	\item Разработана основная часть системы для ручного анализа вредоносных подграфов.
\end{enumerate}
\end{frame}

\begin{frame}
\frametitle{Планы на будущее}
\begin{enumerate}
	\item Продолжать работу над анализом <<плотных>> подграфов, автоматизировать этот процесс за счет дополнительной информации о доменах.
	\item Улучшать текущий анализатор за счет более тонкой настройки на этапе подбора параметров.
	
\end{enumerate}
\end{frame}


\begin{frame}
\frametitle{-}
\Large\centeringСпасибо за внимание!
\end{frame}
\end{document}